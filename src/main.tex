\documentclass{article}
% Document geometry
\usepackage[letterpaper,tmargin=0.5in,
  lmargin=1in,rmargin=1in,bmargin=0.4in,headheight=1in,includehead, includefoot]{geometry}

% For using utf8 with "pdflatex" compilation command
\usepackage[T1]{fontenc}
\usepackage[utf8]{inputenc}
\usepackage{titlesec}
\usepackage{fancyhdr}
\usepackage{enumitem}
\usepackage[page]{totalcount}
\usepackage{hyperref} % Links
\usepackage{pdfpages}

% Header and Footer 
\pagestyle{fancy}
\fancyhf{}
\lhead{\LARGE{\textbf{Bear Valley Solar Development}\\\normalsize{\textbf{Site Visit:} July 22, 2020 1:00 - 3:00 pm}}}
\renewcommand{\headrulewidth}{0pt}
\renewcommand{\footrulewidth}{0pt}
\rfoot{\noindent\includegraphics[height=0.3in]{./branding/weston_big.png}\\\footnotesize{Page \thepage}}
\lfoot{\footnotesize{Created: \today}~HT}
\setlength{\parindent}{0pt}
\setlength{\parskip}{1em}

%\geometry{left=3cm,right=3cm} % 3cm for the sections
\titleformat{\section}% Change section style
{\normalfont\Large\bfseries}% font (normal)
{\hspace{-5em}\thesection}% Label - add some negative horizontal space before
{1em}% Space between label and title
{}
[{\titlerule[0.8pt]}]% Additional code (unused)
% our package:
\usepackage{collage}

\newenvironment{pres}[2]{%
  \large{#2}
  \smallskip
}
{
  \goodbreak
  \smallskip
}

\begin{document}
\section*{Attendees}
\begin{table}[h]
  \label{tab:label}
  \begin{tabular}{l l l l}
      Weston & Hans Tremmel & Project Engineer & 915-731-1274 \\
      BBARWA & John Shimmin & Operations Manager & 760-808-1256\\
  \end{tabular}
\end{table}

\section*{Summary}
The purpose of this field visit was to support 1D and 2D hydrologic modeling.  The following items were investigated during the field visit:
\begin{enumerate}[noitemsep]
    \item Pictures and documentation of outlets, drains.
    \item Pictures and document surrounding topology.  Emphasis on upstream areas.
    \item Note soils conditions
    \item Note sensitive or priority equipment, elevations, foundations, etc.
\end{enumerate}

The operator, John Shimmin, reported that flooding at Big Bear Area Regional Wastewater Agency (BBARWA) wastewater treatment plant is unlikely.  BBARWA was constructed between 15-20 feet above the flood plain (Baldwin Lake). The geology of site consists of a clay layer that prevents most water from infiltrating.

Mr. Shimmin also reported that drains rarely discharge from the BBARWA.  Identified outlets were generally not maintained.  See the attached Site Map for more information.

Visible cracking of the clay layer was observed attesting to the clayey soils.  For the purposes of 2D modeling, the immediate area would likely experience very little infiltration.  The gentle grade of Baldwin Lake will likely experience little scouring as also shown in the attached photos.

The expansive nature of the clay layer may provide challenges to foundation design of the solar development.

\section*{Attachments}
\begin{enumerate}[noitemsep]
    \item Site Map
\end{enumerate}

\includepdf{./0112_200724212622_001.pdf}



\section{Site 1: North of W Slauson Ave between S. Mullen Ave and Kenniston Av}
\refstepcounter{imsection}

\noindent\begin{minipage}{\linewidth}
\begin{InsertImages}\InsertRowOfFigures{\linewidth}{3.1in}{2.5in}{m}{./imgs/Site 1: North of W Slauson Ave between S. Mullen Ave and Kenniston Ave/Green Alleys-2020-07-17-11-53-56.jpg,./imgs/Site 1: North of W Slauson Ave between S. Mullen Ave and Kenniston Ave/Green Alleys-2020-07-17-11-54-03.jpg}
\InsertCaptions[Fig:\theimsection.1,Fig:\theimsection.2]{Green Alleys-2020-07-17-11-53-56,Green Alleys-2020-07-17-11-54-03}{t}{figure}
\InsertRowOfFigures{\linewidth}{3.1in}{2.5in}{m}{./imgs/Site 1: North of W Slauson Ave between S. Mullen Ave and Kenniston Ave/Green Alleys-2020-07-17-11-54-21.jpg,./imgs/Site 1: North of W Slauson Ave between S. Mullen Ave and Kenniston Ave/Green Alleys-2020-07-17-11-56-36.jpg}
\InsertCaptions[Fig:\theimsection.3,Fig:\theimsection.4]{Green Alleys-2020-07-17-11-54-21,Green Alleys-2020-07-17-11-56-36}{t}{figure}
\end{InsertImages}
\end{minipage}


\begin{minipage}{\linewidth}
\begin{InsertImages}
\InsertRowOfFigures{\linewidth}{3.1in}{2.5in}{m}{./imgs/Site 1: North of W Slauson Ave between S. Mullen Ave and Kenniston Ave/Green Alleys-2020-07-17-11-56-46.jpg,./imgs/Site 1: North of W Slauson Ave between S. Mullen Ave and Kenniston Ave/Green Alleys-2020-07-17-11-56-58.jpg}
\InsertCaptions[Fig:\theimsection.5,Fig:\theimsection.6]{Green Alleys-2020-07-17-11-56-46,Green Alleys-2020-07-17-11-56-58}{t}{figure}
\InsertRowOfFigures{\linewidth}{3.1in}{2.5in}{m}{./imgs/Site 1: North of W Slauson Ave between S. Mullen Ave and Kenniston Ave/Green Alleys-2020-07-17-11-57-02.jpg,./imgs/Site 1: North of W Slauson Ave between S. Mullen Ave and Kenniston Ave/Green Alleys-2020-07-17-11-57-51.jpg}
\InsertCaptions[Fig:\theimsection.7,Fig:\theimsection.8]{Green Alleys-2020-07-17-11-57-02,Green Alleys-2020-07-17-11-57-51}{t}{figure}
\end{InsertImages}
\end{minipage}


\begin{minipage}{\linewidth}
\begin{InsertImages}
\InsertRowOfFigures{\linewidth}{3.1in}{2.5in}{m}{./imgs/Site 1: North of W Slauson Ave between S. Mullen Ave and Kenniston Ave/Green Alleys-2020-07-17-11-58-00.jpg,./imgs/Site 1: North of W Slauson Ave between S. Mullen Ave and Kenniston Ave/Green Alleys-2020-07-17-11-59-00.jpg}
\InsertCaptions[Fig:\theimsection.9,Fig:\theimsection.10]{Green Alleys-2020-07-17-11-58-00,Green Alleys-2020-07-17-11-59-00}{t}{figure}
\InsertRowOfFigures{\linewidth}{3.1in}{2.5in}{m}{./imgs/Site 1: North of W Slauson Ave between S. Mullen Ave and Kenniston Ave/Green Alleys-2020-07-17-11-59-11.jpg,./imgs/Site 1: North of W Slauson Ave between S. Mullen Ave and Kenniston Ave/Green Alleys-2020-07-17-12-00-48.jpg}
\InsertCaptions[Fig:\theimsection.11,Fig:\theimsection.12]{Green Alleys-2020-07-17-11-59-11,Green Alleys-2020-07-17-12-00-48}{t}{figure}
\end{InsertImages}
\end{minipage}


\begin{minipage}{\linewidth}
\begin{InsertImages}
\InsertRowOfFigures{\linewidth}{3.1in}{2.5in}{m}{./imgs/Site 1: North of W Slauson Ave between S. Mullen Ave and Kenniston Ave/Green Alleys-2020-07-17-12-00-56.jpg,./imgs/Site 1: North of W Slauson Ave between S. Mullen Ave and Kenniston Ave/Green Alleys-2020-07-17-12-01-59.jpg}
\InsertCaptions[Fig:\theimsection.13,Fig:\theimsection.14]{Green Alleys-2020-07-17-12-00-56,Green Alleys-2020-07-17-12-01-59}{t}{figure}
\InsertRowOfFigures{\linewidth}{3.1in}{2.5in}{m}{./imgs/Site 1: North of W Slauson Ave between S. Mullen Ave and Kenniston Ave/Green Alleys-2020-07-17-12-02-14.jpg,./imgs/Site 1: North of W Slauson Ave between S. Mullen Ave and Kenniston Ave/Green Alleys-2020-07-17-12-03-26.jpg}
\InsertCaptions[Fig:\theimsection.15,Fig:\theimsection.16]{Green Alleys-2020-07-17-12-02-14,Green Alleys-2020-07-17-12-03-26}{t}{figure}
\end{InsertImages}
\end{minipage}


\begin{minipage}{\linewidth}
\begin{InsertImages}
\InsertRowOfFigures{\linewidth}{3.1in}{2.5in}{m}{./imgs/Site 1: North of W Slauson Ave between S. Mullen Ave and Kenniston Ave/Green Alleys-2020-07-17-12-03-34.jpg,./imgs/Site 1: North of W Slauson Ave between S. Mullen Ave and Kenniston Ave/Green Alleys-2020-07-17-12-04-43.jpg}
\InsertCaptions[Fig:\theimsection.17,Fig:\theimsection.18]{Green Alleys-2020-07-17-12-03-34,Green Alleys-2020-07-17-12-04-43}{t}{figure}
\InsertRowOfFigures{\linewidth}{3.1in}{2.5in}{m}{./imgs/Site 1: North of W Slauson Ave between S. Mullen Ave and Kenniston Ave/Green Alleys-2020-07-17-12-04-58.jpg,./imgs/Site 1: North of W Slauson Ave between S. Mullen Ave and Kenniston Ave/Green Alleys-2020-07-17-12-05-07.jpg}
\InsertCaptions[Fig:\theimsection.19,Fig:\theimsection.20]{Green Alleys-2020-07-17-12-04-58,Green Alleys-2020-07-17-12-05-07}{t}{figure}
\end{InsertImages}
\end{minipage}


\begin{minipage}{\linewidth}
\begin{InsertImages}
\InsertRowOfFigures{\linewidth}{3.1in}{2.5in}{m}{./imgs/Site 1: North of W Slauson Ave between S. Mullen Ave and Kenniston Ave/Green Alleys-2020-07-17-12-05-14.jpg,./imgs/Site 1: North of W Slauson Ave between S. Mullen Ave and Kenniston Ave/Green Alleys-2020-07-17-12-06-07.jpg}
\InsertCaptions[Fig:\theimsection.21,Fig:\theimsection.22]{Green Alleys-2020-07-17-12-05-14,Green Alleys-2020-07-17-12-06-07}{t}{figure}
\InsertRowOfFigures{\linewidth}{3.1in}{2.5in}{m}{./imgs/Site 1: North of W Slauson Ave between S. Mullen Ave and Kenniston Ave/Green Alleys-2020-07-17-12-06-13.jpg,./imgs/Site 1: North of W Slauson Ave between S. Mullen Ave and Kenniston Ave/Green Alleys-2020-07-17-12-07-02.jpg}
\InsertCaptions[Fig:\theimsection.23,Fig:\theimsection.24]{Green Alleys-2020-07-17-12-06-13,Green Alleys-2020-07-17-12-07-02}{t}{figure}
\end{InsertImages}
\end{minipage}


\begin{minipage}{\linewidth}
\begin{InsertImages}
\InsertRowOfFigures{\linewidth}{3.1in}{2.5in}{m}{./imgs/Site 1: North of W Slauson Ave between S. Mullen Ave and Kenniston Ave/Green Alleys-2020-07-17-12-08-37.jpg,./imgs/Site 1: North of W Slauson Ave between S. Mullen Ave and Kenniston Ave/Green Alleys-2020-07-17-12-08-44.jpg}
\InsertCaptions[Fig:\theimsection.25,Fig:\theimsection.26]{Green Alleys-2020-07-17-12-08-37,Green Alleys-2020-07-17-12-08-44}{t}{figure}
\InsertRowOfFigures{\linewidth}{3.1in}{2.5in}{m}{./imgs/Site 1: North of W Slauson Ave between S. Mullen Ave and Kenniston Ave/Green Alleys-2020-07-17-12-08-55.jpg,./imgs/Site 1: North of W Slauson Ave between S. Mullen Ave and Kenniston Ave/Green Alleys-2020-07-17-12-08-58.jpg}
\InsertCaptions[Fig:\theimsection.27,Fig:\theimsection.28]{Green Alleys-2020-07-17-12-08-55,Green Alleys-2020-07-17-12-08-58}{t}{figure}
\end{InsertImages}
\end{minipage}


\begin{minipage}{\linewidth}
\begin{InsertImages}
\InsertRowOfFigures{\linewidth}{3.1in}{2.5in}{m}{./imgs/Site 1: North of W Slauson Ave between S. Mullen Ave and Kenniston Ave/Green Alleys-2020-07-17-12-10-00.jpg,./imgs/Site 1: North of W Slauson Ave between S. Mullen Ave and Kenniston Ave/Green Alleys-2020-07-17-12-10-10.jpg}
\InsertCaptions[Fig:\theimsection.29,Fig:\theimsection.30]{Green Alleys-2020-07-17-12-10-00,Green Alleys-2020-07-17-12-10-10}{t}{figure}
\InsertRowOfFigures{\linewidth}{3.1in}{2.5in}{m}{./imgs/Site 1: North of W Slauson Ave between S. Mullen Ave and Kenniston Ave/Green Alleys-2020-07-17-12-11-28.jpg,./imgs/Site 1: North of W Slauson Ave between S. Mullen Ave and Kenniston Ave/Green Alleys-2020-07-17-12-18-01.jpg}
\InsertCaptions[Fig:\theimsection.31,Fig:\theimsection.32]{Green Alleys-2020-07-17-12-11-28,Green Alleys-2020-07-17-12-18-01}{t}{figure}
\end{InsertImages}
\end{minipage}


\begin{minipage}{\linewidth}
\begin{InsertImages}
\InsertRowOfFigures{\linewidth}{3.1in}{2.5in}{m}{./imgs/Site 1: North of W Slauson Ave between S. Mullen Ave and Kenniston Ave/Green Alleys-2020-07-17-12-18-16.jpg,./imgs/Site 1: North of W Slauson Ave between S. Mullen Ave and Kenniston Ave/Green Alleys-2020-07-17-12-18-25.jpg}
\InsertCaptions[Fig:\theimsection.33,Fig:\theimsection.34]{Green Alleys-2020-07-17-12-18-16,Green Alleys-2020-07-17-12-18-25}{t}{figure}
\InsertRowOfFigures{\linewidth}{3.1in}{2.5in}{m}{./imgs/Site 1: North of W Slauson Ave between S. Mullen Ave and Kenniston Ave/Green Alleys-2020-07-17-12-18-33.jpg,./imgs/Site 1: North of W Slauson Ave between S. Mullen Ave and Kenniston Ave/Green Alleys-2020-07-17-12-19-26.jpg}
\InsertCaptions[Fig:\theimsection.35,Fig:\theimsection.36]{Green Alleys-2020-07-17-12-18-33,Green Alleys-2020-07-17-12-19-26}{t}{figure}
\end{InsertImages}
\end{minipage}


\begin{minipage}{\linewidth}
\begin{InsertImages}
\InsertRowOfFigures{\linewidth}{3.1in}{2.5in}{m}{./imgs/Site 1: North of W Slauson Ave between S. Mullen Ave and Kenniston Ave/Green Alleys-2020-07-17-12-19-36.jpg}
\InsertCaptions[Fig:\theimsection.37]{Green Alleys-2020-07-17-12-19-36}{t}{figure}
\end{InsertImages}
\end{minipage}



\section{Site 3: North of Imperial hwy between S. Hobart Blvd and S. Normandie Av}
\refstepcounter{imsection}

\noindent\begin{minipage}{\linewidth}
\begin{InsertImages}\InsertRowOfFigures{\linewidth}{3.1in}{2.5in}{m}{./imgs/Site 3: North of Imperial hwy between S. Hobart Blvd and S. Normandie Ave/Green Alleys-2020-07-17-11-20-44.jpg,./imgs/Site 3: North of Imperial hwy between S. Hobart Blvd and S. Normandie Ave/Green Alleys-2020-07-17-11-21-05.jpg}
\InsertCaptions[Fig:\theimsection.1,Fig:\theimsection.2]{Green Alleys-2020-07-17-11-20-44,Green Alleys-2020-07-17-11-21-05}{t}{figure}
\InsertRowOfFigures{\linewidth}{3.1in}{2.5in}{m}{./imgs/Site 3: North of Imperial hwy between S. Hobart Blvd and S. Normandie Ave/Green Alleys-2020-07-17-11-21-18.jpg,./imgs/Site 3: North of Imperial hwy between S. Hobart Blvd and S. Normandie Ave/Green Alleys-2020-07-17-11-21-21.jpg}
\InsertCaptions[Fig:\theimsection.3,Fig:\theimsection.4]{Green Alleys-2020-07-17-11-21-18,Green Alleys-2020-07-17-11-21-21}{t}{figure}
\end{InsertImages}
\end{minipage}


\begin{minipage}{\linewidth}
\begin{InsertImages}
\InsertRowOfFigures{\linewidth}{3.1in}{2.5in}{m}{./imgs/Site 3: North of Imperial hwy between S. Hobart Blvd and S. Normandie Ave/Green Alleys-2020-07-17-11-23-43.jpg,./imgs/Site 3: North of Imperial hwy between S. Hobart Blvd and S. Normandie Ave/Green Alleys-2020-07-17-11-23-50.jpg}
\InsertCaptions[Fig:\theimsection.5,Fig:\theimsection.6]{Green Alleys-2020-07-17-11-23-43,Green Alleys-2020-07-17-11-23-50}{t}{figure}
\InsertRowOfFigures{\linewidth}{3.1in}{2.5in}{m}{./imgs/Site 3: North of Imperial hwy between S. Hobart Blvd and S. Normandie Ave/Green Alleys-2020-07-17-11-24-04.jpg,./imgs/Site 3: North of Imperial hwy between S. Hobart Blvd and S. Normandie Ave/Green Alleys-2020-07-17-11-25-21.jpg}
\InsertCaptions[Fig:\theimsection.7,Fig:\theimsection.8]{Green Alleys-2020-07-17-11-24-04,Green Alleys-2020-07-17-11-25-21}{t}{figure}
\end{InsertImages}
\end{minipage}


\begin{minipage}{\linewidth}
\begin{InsertImages}
\InsertRowOfFigures{\linewidth}{3.1in}{2.5in}{m}{./imgs/Site 3: North of Imperial hwy between S. Hobart Blvd and S. Normandie Ave/Green Alleys-2020-07-17-11-25-35.jpg,./imgs/Site 3: North of Imperial hwy between S. Hobart Blvd and S. Normandie Ave/Green Alleys-2020-07-17-11-25-48.jpg}
\InsertCaptions[Fig:\theimsection.9,Fig:\theimsection.10]{Green Alleys-2020-07-17-11-25-35,Green Alleys-2020-07-17-11-25-48}{t}{figure}
\InsertRowOfFigures{\linewidth}{3.1in}{2.5in}{m}{./imgs/Site 3: North of Imperial hwy between S. Hobart Blvd and S. Normandie Ave/Green Alleys-2020-07-17-11-25-51.jpg,./imgs/Site 3: North of Imperial hwy between S. Hobart Blvd and S. Normandie Ave/Green Alleys-2020-07-17-11-25-55.jpg}
\InsertCaptions[Fig:\theimsection.11,Fig:\theimsection.12]{Green Alleys-2020-07-17-11-25-51,Green Alleys-2020-07-17-11-25-55}{t}{figure}
\end{InsertImages}
\end{minipage}


\begin{minipage}{\linewidth}
\begin{InsertImages}
\InsertRowOfFigures{\linewidth}{3.1in}{2.5in}{m}{./imgs/Site 3: North of Imperial hwy between S. Hobart Blvd and S. Normandie Ave/Green Alleys-2020-07-17-11-27-59.jpg,./imgs/Site 3: North of Imperial hwy between S. Hobart Blvd and S. Normandie Ave/Green Alleys-2020-07-17-11-28-13.jpg}
\InsertCaptions[Fig:\theimsection.13,Fig:\theimsection.14]{Green Alleys-2020-07-17-11-27-59,Green Alleys-2020-07-17-11-28-13}{t}{figure}
\InsertRowOfFigures{\linewidth}{3.1in}{2.5in}{m}{./imgs/Site 3: North of Imperial hwy between S. Hobart Blvd and S. Normandie Ave/Green Alleys-2020-07-17-11-29-22.jpg,./imgs/Site 3: North of Imperial hwy between S. Hobart Blvd and S. Normandie Ave/Green Alleys-2020-07-17-11-29-31.jpg}
\InsertCaptions[Fig:\theimsection.15,Fig:\theimsection.16]{Green Alleys-2020-07-17-11-29-22,Green Alleys-2020-07-17-11-29-31}{t}{figure}
\end{InsertImages}
\end{minipage}


\begin{minipage}{\linewidth}
\begin{InsertImages}
\InsertRowOfFigures{\linewidth}{3.1in}{2.5in}{m}{./imgs/Site 3: North of Imperial hwy between S. Hobart Blvd and S. Normandie Ave/Green Alleys-2020-07-17-11-31-22.jpg,./imgs/Site 3: North of Imperial hwy between S. Hobart Blvd and S. Normandie Ave/Green Alleys-2020-07-17-11-31-29.jpg}
\InsertCaptions[Fig:\theimsection.17,Fig:\theimsection.18]{Green Alleys-2020-07-17-11-31-22,Green Alleys-2020-07-17-11-31-29}{t}{figure}
\InsertRowOfFigures{\linewidth}{3.1in}{2.5in}{m}{./imgs/Site 3: North of Imperial hwy between S. Hobart Blvd and S. Normandie Ave/Green Alleys-2020-07-17-11-31-36.jpg}
\InsertCaptions[Fig:\theimsection.19]{Green Alleys-2020-07-17-11-31-36}{t}{figure}
\end{InsertImages}
\end{minipage}



\section{Site 2: W 54th St between Deane Ave and S. Harcourt Av}
\refstepcounter{imsection}

\noindent\begin{minipage}{\linewidth}
\begin{InsertImages}\InsertRowOfFigures{\linewidth}{3.1in}{2.5in}{m}{./imgs/Site 2: W 54th St between Deane Ave and S. Harcourt Ave/Green Alleys-2020-07-17-12-21-05.jpg,./imgs/Site 2: W 54th St between Deane Ave and S. Harcourt Ave/Green Alleys-2020-07-17-12-21-18.jpg}
\InsertCaptions[Fig:\theimsection.1,Fig:\theimsection.2]{Green Alleys-2020-07-17-12-21-05,Green Alleys-2020-07-17-12-21-18}{t}{figure}
\InsertRowOfFigures{\linewidth}{3.1in}{2.5in}{m}{./imgs/Site 2: W 54th St between Deane Ave and S. Harcourt Ave/Green Alleys-2020-07-17-12-21-30.jpg,./imgs/Site 2: W 54th St between Deane Ave and S. Harcourt Ave/Green Alleys-2020-07-17-12-23-04.jpg}
\InsertCaptions[Fig:\theimsection.3,Fig:\theimsection.4]{Green Alleys-2020-07-17-12-21-30,Green Alleys-2020-07-17-12-23-04}{t}{figure}
\end{InsertImages}
\end{minipage}


\begin{minipage}{\linewidth}
\begin{InsertImages}
\InsertRowOfFigures{\linewidth}{3.1in}{2.5in}{m}{./imgs/Site 2: W 54th St between Deane Ave and S. Harcourt Ave/Green Alleys-2020-07-17-12-23-13.jpg,./imgs/Site 2: W 54th St between Deane Ave and S. Harcourt Ave/Green Alleys-2020-07-17-12-24-18.jpg}
\InsertCaptions[Fig:\theimsection.5,Fig:\theimsection.6]{Green Alleys-2020-07-17-12-23-13,Green Alleys-2020-07-17-12-24-18}{t}{figure}
\InsertRowOfFigures{\linewidth}{3.1in}{2.5in}{m}{./imgs/Site 2: W 54th St between Deane Ave and S. Harcourt Ave/Green Alleys-2020-07-17-12-24-26.jpg,./imgs/Site 2: W 54th St between Deane Ave and S. Harcourt Ave/Green Alleys-2020-07-17-12-24-30.jpg}
\InsertCaptions[Fig:\theimsection.7,Fig:\theimsection.8]{Green Alleys-2020-07-17-12-24-26,Green Alleys-2020-07-17-12-24-30}{t}{figure}
\end{InsertImages}
\end{minipage}


\begin{minipage}{\linewidth}
\begin{InsertImages}
\InsertRowOfFigures{\linewidth}{3.1in}{2.5in}{m}{./imgs/Site 2: W 54th St between Deane Ave and S. Harcourt Ave/Green Alleys-2020-07-17-12-24-41.jpg,./imgs/Site 2: W 54th St between Deane Ave and S. Harcourt Ave/Green Alleys-2020-07-17-12-25-59.jpg}
\InsertCaptions[Fig:\theimsection.9,Fig:\theimsection.10]{Green Alleys-2020-07-17-12-24-41,Green Alleys-2020-07-17-12-25-59}{t}{figure}
\InsertRowOfFigures{\linewidth}{3.1in}{2.5in}{m}{./imgs/Site 2: W 54th St between Deane Ave and S. Harcourt Ave/Green Alleys-2020-07-17-12-26-15.jpg,./imgs/Site 2: W 54th St between Deane Ave and S. Harcourt Ave/Green Alleys-2020-07-17-12-26-24.jpg}
\InsertCaptions[Fig:\theimsection.11,Fig:\theimsection.12]{Green Alleys-2020-07-17-12-26-15,Green Alleys-2020-07-17-12-26-24}{t}{figure}
\end{InsertImages}
\end{minipage}


\begin{minipage}{\linewidth}
\begin{InsertImages}
\InsertRowOfFigures{\linewidth}{3.1in}{2.5in}{m}{./imgs/Site 2: W 54th St between Deane Ave and S. Harcourt Ave/Green Alleys-2020-07-17-12-26-28.jpg,./imgs/Site 2: W 54th St between Deane Ave and S. Harcourt Ave/Green Alleys-2020-07-17-12-27-11.jpg}
\InsertCaptions[Fig:\theimsection.13,Fig:\theimsection.14]{Green Alleys-2020-07-17-12-26-28,Green Alleys-2020-07-17-12-27-11}{t}{figure}
\InsertRowOfFigures{\linewidth}{3.1in}{2.5in}{m}{./imgs/Site 2: W 54th St between Deane Ave and S. Harcourt Ave/Green Alleys-2020-07-17-12-27-20.jpg,./imgs/Site 2: W 54th St between Deane Ave and S. Harcourt Ave/Green Alleys-2020-07-17-12-28-05.jpg}
\InsertCaptions[Fig:\theimsection.15,Fig:\theimsection.16]{Green Alleys-2020-07-17-12-27-20,Green Alleys-2020-07-17-12-28-05}{t}{figure}
\end{InsertImages}
\end{minipage}


\begin{minipage}{\linewidth}
\begin{InsertImages}
\InsertRowOfFigures{\linewidth}{3.1in}{2.5in}{m}{./imgs/Site 2: W 54th St between Deane Ave and S. Harcourt Ave/Green Alleys-2020-07-17-12-28-22.jpg,./imgs/Site 2: W 54th St between Deane Ave and S. Harcourt Ave/Green Alleys-2020-07-17-12-28-29.jpg}
\InsertCaptions[Fig:\theimsection.17,Fig:\theimsection.18]{Green Alleys-2020-07-17-12-28-22,Green Alleys-2020-07-17-12-28-29}{t}{figure}
\InsertRowOfFigures{\linewidth}{3.1in}{2.5in}{m}{./imgs/Site 2: W 54th St between Deane Ave and S. Harcourt Ave/Green Alleys-2020-07-17-12-28-33.jpg}
\InsertCaptions[Fig:\theimsection.19]{Green Alleys-2020-07-17-12-28-33}{t}{figure}
\end{InsertImages}
\end{minipage}





\end{document}
